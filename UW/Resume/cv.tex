%% start of file `template.tex'.
%% Copyright 2006-2013 Xavier Danaux (xdanaux@gmail.com).
%
% This work may be distributed and/or modified under the
% conditions of the LaTeX Project Public License version 1.3c,
% available at http://www.latex-project.org/lppl/.


\documentclass[11pt,a4paper,sans]{moderncv}        % possible options include font size ('10pt', '11pt' and '12pt'), paper size ('a4paper', 'letterpaper', 'a5paper', 'legalpaper', 'executivepaper' and 'landscape') and font family ('sans' and 'roman')

% moderncv themes
\moderncvstyle{classic}                            % style options are 'casual' (default), 'classic', 'oldstyle' and 'banking'
\moderncvcolor{blue}                               % color options 'blue' (default), 'orange', 'green', 'red', 'purple', 'grey' and 'black'
%\renewcommand{\familydefault}{\sfdefault}         % to set the default font; use '\sfdefault' for the default sans serif font, '\rmdefault' for the default roman one, or any tex font name
%\nopagenumbers{}                                  % uncomment to suppress automatic page numbering for CVs longer than one page

% character encoding
%\usepackage[utf8]{inputenc}                       % if you are not using xelatex ou lualatex, replace by the encoding you are using
%\usepackage{CJKutf8}                              % if you need to use CJK to typeset your resume in Chinese, Japanese or Korean

% adjust the page margins
\usepackage[scale=0.8]{geometry}
%\setlength{\hintscolumnwidth}{3cm}                % if you want to change the width of the column with the dates
%\setlength{\makecvtitlenamewidth}{10cm}           % for the 'classic' style, if you want to force the width allocated to your name and avoid line breaks. be careful though, the length is normally calculated to avoid any overlap with your personal info; use this at your own typographical risks...

% personal data
\name{John}{Wright}
%\title{Resumé title}
\address{770 N. 73rd Street}{Seattle, WA 98103}{}
\phone[mobile]{+1~(206)~795~8982}
%\phone[fixed]{+2~(345)~678~901}
%\phone[fax]{+3~(456)~789~012}
\email{johnwri@uw.edu}
\homepage{www.pymud.net}
%\social[linkedin]{john.doe}
%\social[twitter]{jdoe}
%\social[github]{jdoe}
%\extrainfo{additional information}
%\photo[64pt][0.4pt]{closeup.jpg}                       % optional, remove / comment the line if not wanted; '64pt' is the height the picture must be resized to, 0.4pt is the thickness of the frame around it (put it to 0pt for no frame) and 'picture' is the name of the picture file

% to show numerical labels in the bibliography (default is to show no labels); only useful if you make citations in your resume
%\makeatletter
%\renewcommand*{\bibliographyitemlabel}{\@biblabel{\arabic{enumiv}}}
%\makeatother
%\renewcommand*{\bibliographyitemlabel}{[\arabic{enumiv}]}% CONSIDER REPLACING THE ABOVE BY THIS

% bibliography with mutiple entries
%\usepackage{multibib}
%\newcites{book,misc}{{Books},{Others}}

%-------------------------------------------------------------------------------
%            content
%-------------------------------------------------------------------------------
\begin{document}
%-----       resume       ------------------------------------------------------
\makecvtitle

\section{Education}
\cventry{2010--March 2015}{Physics Ph.D.}{University of Washington}{Seattle, WA}{\textit{3.81/4.00}}{Experimental Ion Trapping with Applications in Quantum Computation}
\cventry{2006--2010}{Bachelor of Science}{Purdue University}{West Lafayette, IN}{\textit{3.96/4.00}}{Majored in Physics, Computer Science, and Math}

\section{Work Experience}
\cventry{2010--Present}{Research Assistant}{University of Washington}{Seattle, WA}{}{Built an UHV ion trapping apparatus for  evaluating quantum computer architectures using Barium and Ytterbium ions.
	\begin{itemize}
		\item Aligned entire optical setup including fast power, frequency, and polarization control of ionization, cooling, and quantum operations lasers for two ion species;
		\item Simulated ion dynamics, ion trapping potentials, and optical systems using custom and commercial programs;
		\item Designed and fabricated PCBs with embedded microcontrollers and FPGAs for stabilizing lasers and controlling trap voltages;
		\item Assembled and operated a CF UHV chamber that consistently reached pressures of $5\times10^{-11}$ torr.
		\item Met grant objectives on time and within budget;
		\item Supervised several undergraduate and junior graduate students while doing research, taking graduate courses, and teaching undergraduates.
	\end{itemize}}
\cventry{2006--2010}{Undergraduate Research Assistant}{Purdue University}{West Lafayette, IN}{}{Wrote software for simulation and data analysis of the Very Energetic Radiation Imaging Telescope Array System and Large Synoptic Survey Telescope.
	\begin{itemize}
		\item Automated simulations of gamma ray showers on medium size computing clusters, modified simulation code to track all gamma ray shower particles, and developed visualization software for demonstration and debugging;
		\item Wrote image analysis routines for tracking sources in images from the Large Synoptic Survey Telescope.
	\end{itemize}}

\section{Technical Skills}
\cvitem{Design/Analysis Software}{OpticStudio, Quartus, LabView, Visual Studio, Mathematica, MatLab, GSL, PETSc, MPI, OpenCV, OpenGL}
\cvitem{Embedded Systems}{NI-DAQ, Altera FPGAs (Cyclone series), Analog Devices Microcontrollers (ARM7 CPUs), Analog Devices DACs and DDSs, Arduino}
\cvitem{Programming}{C, C++, Java, Python, C\#, Verilog, Haskell, Javascript}
%\cvitem{Test Equipment}{Optical/RF Spectrum Analyzers and Power Meters, Oscilloscopes, Multimeters, LCR Meters}
\cvitem{Optical Systems}{Acousto-optic Modulators, Electro-optic Modulators, Pockels Cells, Single Mode Fibers, External Cavity Diode Lasers, Nd:YAG Lasers, Ti:Sapphire Lasers, Fiber Lasers}

%\section{Skills}
%\cvitem{Embedded Control}{Implemented embedded control systems using Analog Devices ARM7 CPUs (ADuC7020 family) and simple Arduino microcontrollers.  Built PWM temperature controllers, stepper motor drivers, and piezoelectric actuator drivers with serial interfaces.}
%\cvitem{FPGA/Fast Control}{Automated complicated experiments with 100 MHz timing requirements using both National Instruments technologies and purpose-built FPGA solutions.  Wrote Verilog programs for photon counting and fast voltage control with an Ethernet interface using ARP and UDP.}
%\cvitem{Physics Simulation}{Performed electrostatic modeling of trap geometries as well as ion molecular dynamics simulations of trap stability.  Worked with parallel algorithms using free, open-source tools to rapidly evaluate trapping strategies.}
%\cvitem{Electronics Prototyping}{Designed, printed, populated, and tested custom PCBs for temperature control, laser frequency stabilization, and fast many-channel DACs with embedded control systems.}
%\cvitem{Laser Design}{Designed, machined, and aligned external-cavity, diode lasers used for ionization and cooling and a solid state diode-pumped mode-locked Nd:YAG laser used for quantum operations.  Reconfigured and aligned Ti:Sapphire mode-locked lasers.}
%\cvitem{Optical Alignment}{Coupled and optimized various optical systems including acousto-optic modulators, electro-optic modulators, Pockels cells, single-mode fiber couplers, optical cavities, and optical isolators.}
%\cvitem{Vacuum Systems}{Assembled and operated several CF UHV vacuum systems with ion and titanium sublimation pumps that consistently reached $5\times10^{-11}$ torr.}

\section{Interests}
\cvitem{Graphics Programming}{Used OpenGL and WebGL to implement physically accurate graphical simulations of space scenes and to view the results of electrostatic simulations.}
\cvitem{Machine Learning}{Wrote custom speech recognition system using hidden markov models to preform phoneme alignment and train triphone Gaussian mixture models to recognize phonemes.}
\cvitem{Science Outreach}{Volunteered with local events including Dawg Days and Paws on Science to demonstrate models of ion trapping and quantum information to children and the general public.}

\section{Awards}
\cvitem{Miller Award}{Graduate student award for need, excellence of character, and scholarship.}
\cvitem{Purdue Presidential Scholarship}{Recognizes high academic achievement with experience in leadership, service, and/or community activity.}

\renewcommand{\refname}{Publications}
\nocite{*}
\bibliographystyle{plain}
\bibliography{publications}

\clearpage
%-----       letter       ------------------------------------------------------
% recipient data
%\recipient{Company Recruitment team}{Company, Inc.\\123 somestreet\\some city}
%\date{January 01, 1984}
%\opening{Dear Sir or Madam,}
%\closing{Yours faithfully,}
%\enclosure[Attached]{curriculum vit\ae{}}
%\makelettertitle

%	Lorem ipsum dolor sit amet, consectetur adipiscing elit. Duis ullamcorper neque sit amet lectus facilisis sed luctus nisl iaculis. Vivamus at neque arcu, sed tempor quam. Curabitur pharetra %tincidunt tincidunt. Morbi volutpat feugiat mauris, quis tempor neque vehicula volutpat. Duis tristique justo vel massa fermentum accumsan. Mauris ante elit, feugiat vestibulum tempor eget, %eleifend ac ipsum. Donec scelerisque lobortis ipsum eu vestibulum. Pellentesque vel massa at felis accumsan rhoncus.

%	Suspendisse commodo, massa eu congue tincidunt, elit mauris pellentesque orci, cursus tempor odio nisl euismod augue. Aliquam adipiscing nibh ut odio sodales et pulvinar tortor %laoreet. Mauris a accumsan ligula. Class aptent taciti sociosqu ad litora torquent per conubia nostra, per inceptos himenaeos. Suspendisse vulputate sem vehicula ipsum varius nec tempus %dui dapibus. Phasellus et est urna, ut auctor erat. Sed tincidunt odio id odio aliquam mattis. Donec sapien nulla, feugiat eget adipiscing sit amet, lacinia ut dolor. Phasellus tincidunt, leo a %fringilla consectetur, felis diam aliquam urna, vitae aliquet lectus orci nec velit. Vivamus dapibus varius blandit.

%	Duis sit amet magna ante, at sodales diam. Aenean consectetur porta risus et sagittis. Ut interdum, enim varius pellentesque tincidunt, magna libero sodales tortor, ut fermentum nunc %metus a ante. Vivamus odio leo, tincidunt eu luctus ut, sollicitudin sit amet metus. Nunc sed orci lectus. Ut sodales magna sed velit volutpat sit amet pulvinar diam venenatis.

%	Albert Einstein discovered that $e=mc^2$ in 1905.

%	\[ e=\lim_{n \to \infty} \left(1+\frac{1}{n}\right)^n \]

%\makeletterclosing
\end{document}


