Mixed species chains of barium and ytterbium ions are investigated as a tool for building scalable quantum computation devices.  Ytterbium ions provide a stable, environmentally-insensitive qubit that is easily initialized and manipulated, while barium ions are easily entangled with photons that can allow quantum information to be transmitted between systems in modular quantum computation units.  Barium and ytterbium are trapped together in a linear chain in a linear rf trap and their normal mode structure and the thermal occupation numbers of these modes are measured with a narrow band laser addressing an electric quadrupole transition in barium ions. Before these measurements, barium ions are directly cooled using Doppler cooling, while the ytterbium ions are sympathetically cooled by the barium.  For radial modes strongly coupled to ytterbium ions the average thermal occupation numbers vary between 400 and 12,000 depending on ion species configuration and trap parameters.  Ion chain temperatures are also measured using a technique based on ion species reordering.  Surface traps with many dc electrodes provide the ability to controllably reorder the chain to optimize normal mode cooling, and initial work towards realizing this capability are discussed.  Quantum information can be transferred between ions in a linear chain using an optical system that is well coupled to the motional degrees of freedom of the chain.  For this reason, a 532~nm Raman system is developed and its expected performance is evaluated.
