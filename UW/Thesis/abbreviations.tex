\begin{description}
	\item[ADC] Analog to Digital Converter.  Converts analog voltage or current signals to the digital representation of a corresponding number in Volts or Amperes.

	\item[AOM] Acousto-Optic Modulator.  Device to diffract an incoming laser beam against a rf standing wave.  Allows the frequency and power of a diffracted beam to be quickly controlled by changing RF parameters.

	\item[CW] Continuous Wave.  Mode of laser operation with constant light emission.  Often used to distinguish from pulsed or mode-locked operation, where the laser outputs light in discrete pulses.

	\item[DAC] Digital to Analog Converter.  Converts a digital representation of a number to a corresponding analog signal of that many Volts or Amperes.

	\item[DC] Direct Current.  Current flowing in a constant direction.  Often used to refer to signals with no time variation.

	\item[DDS] Direct Digital Synthesizer.  Frequency synthesizer with digital control interface. 

	\item[DPAOM] Double Passed Acousto-Optic Modulator.  Optical system that diffracts light twice off an AOM to increase optical isolation and minimize pointing shifts due to changing the rf frequency applied to the AOM.

	\item[DSUB] D-subminiature.  A common type of electrical connector often used for serial connections.

	\item[ECDL] External Cavity Diode Laser.  Laser made by externally frequency selecting and feeding back light into a laser diode.

	\item[EMCCD]  Electron Multiplying Charge Coupled Device. Charge coupled device with an electron multiplying readout channel with controllable gain.

	\item[EOM] Electro-Optic Modulator.  Device to modulate a laser beam with an electric control field.  Allows modulation of the frequency, power, or phase of the beam.

	\item[FPGA] Field Programmable Gate Array.  Integrated circuit with programmable interconnects between logical gates enabling rapid development of logic circuits.

	\item[HWP] Half Wave Plate. Delays light polarized along one direction by half a wavelength while not affecting the orthogonal direction of polarization.  Often used to rotate the polarization of light by a desired angle.

	\item[PBS] Polarizing Beam Splitter.  Reflects light polarized along one direction while transmitting the orthogonal direction of polarization.

	\item[PCB] Printed Circuit Board.  Insulating board with small printed conducting traces on its surface.  Often printed to implement complicated circuits without large amounts of tedious wiring.

	\item[PID Controller] Proportional, Integral, Differential Controller.  Feedback controller with three feedback terms proportional to the error signal, its integral, and its derivative.

	\item[PMT] Photomultiplier Tube.  Single photon detection device based on photoelectron amplification.

	\item[QWP] Quarter Wave Plate. Delays light polarized along one direction by a quarter of a wavelength while not affecting the orthogonal direction of polarization.  Often used to elliptically or circularly polarize light.

	\item[RF] Radio Frequency.  Frequencies between a few kHz and $\approx$ 1000~GHz.

	\item[SRAM] Static Random-Access Memory.  Digital, computer memory that is stable so long as power is provided to it.

	\item[TTL] Transistor-Transistor Logic.  Digital signaling standard using voltages between 0 and 0.8~V for ``0'' and 2.2 to 5~V for ``1''.

	\item[UHV] Ultra-High Vacuum.  Pressures below 10$^{-9}$~torr.  Usually only achievable with all metal seals and careful chamber preparation.	

	\item[UDP] User Datagram Protocol.  Protocol for sending information over an Internet Protocol network.  
\end{description}

