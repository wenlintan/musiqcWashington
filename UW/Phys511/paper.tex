\documentclass[11pt]{article}

\usepackage{graphicx}    % needed for including graphics e.g. EPS, PS
\usepackage{braket}
\usepackage{mathtools}

\topmargin -1.5cm        % read Lamport p.163
\oddsidemargin -0.04cm   % read Lamport p.163
\evensidemargin -0.04cm  % same as oddsidemargin but for left-hand pages
\textwidth 16.59cm
\textheight 21.94cm 

%\pagestyle{empty}       % Uncomment if don't want page numbers
\parskip 7.2pt           % sets spacing between paragraphs
%\renewcommand{\baselinestretch}{1.5} 	% Uncomment for 1.5 spacing between lines
\parindent 0pt		 	% sets leading space for paragraphs

\begin{document}         
\title{Detecting and Quantifying Entanglement in Bipartite and Multipartite Systems}
\author{John Wright \\ \emph{Department of Physics, University of Washington, Seattle, Washington 98105, USA}}
\date{\today}
\maketitle

\begin{abstract}
Proposed methods for detecting entanglement include the Clauser-Horne-Shimony-Holt inequality, so called
Geneva and Hefei inequalities, as well as entanglement witnesses.  A set of mixed and pure, entangled and 
separable polarization states are prepared by parametric down conversion and their entanglement is 
classified using these detection methods.  Although these methods sometimes allow detection of entanglement, they 
do not quantify the degree of entanglement and they require at least some knowledge of the initial state 
in order to function.
\end{abstract}

\section{Introduction}
Quantum entanglement allows quantum computation to achieve its remarkable speed increase over all classical
algorithms, but it can be a difficult resource to measure and quantify.  Entangled systems are also being
used to measure quantum decoherence into environmental degrees of freedom and perform quantum eraser
experiments.  In any quantum experiment, having full knowledge of the state of the system, including
couplings caused by noise and the infidelity of quantum operations, is difficult, but necessary in order
to understand these processes.  The obvious solution is quantum state tomography which measures the 
fraction of the population in every state of the density matrix.  However, the size of the density matrix 
grows exponentially with the size of the system and many terms in the density matrix may be exponentially 
suppressed (requiring exponentially many measurements to resolve them).  Measurements also need to be taken
in most of the possible bases of the system.  It is clear that performing state tomography on even
moderately sized quantum systems is currently infeasible.  Instead simple measurements that at least
quantify properties of systems, such as the degree of entanglement, can be made.  Ideally these measurements
should require measurement in fewer bases, which may be experimentally costly.  Originally most
methods for detecting entanglement were based on Bell's inequality and involved measuring the degree of local 
realism violation, but recently new methods based on entanglement witness operators have been developed.
These methods will be experimentally demonstrated using polarization entangled photons.

\section{Experimental Apparatus and State Generation}
A 351nm pump beam is incident on a nonlinear BBO crystal with phase matching conditions set such that horizonatlly 
polarized incoming photons are converted into two vertically polarized photons, immediately followed by a 
similar crystal oriented orthogonally. Therefore an arbitrary incoming state \( \alpha \ket{H} + \beta \ket{V} \) 
is converted into the entangled state \( \alpha \ket{VV} + \beta \ket{HH} \).  The resulting state is described as 
\[ \ket{\psi(\theta, \phi)}  = \cos (\theta) \ket{HH} + \sin (\theta) e^{i\phi} \ket{VV}, \]
where \( \theta \) is set by the initial polarization of the incoming light and \( \phi \) is set by tilting a 
quarter wave plate about its optic axis to selectively retard the vertical polarization state.  Two BBO
crystals, again orthogonally oriented, compensate for phase delays between the photons created at different
locations inside the first crystal\cite{altepeter05}.

In order to generate both mixed and separable states quartz crystals were used as quantum decoherers.
The coherence length of the entangled photons was reduced by introducing interference filters before
the measurement photodiodes and the length of the quartz piece was chosen to be sufficiently long to completely
decohere photons propogating along the extraordinary ray from those propogating along the ordinary
ray.  The quartz crystals are orthogonal, so only photons in the states \( \ket{VV}\bra{HH} \)
and \( \ket{HH}\bra{VV} \) remain coherent.  Note that this decoherence corresponds to entangling the 
polarization state information with frequency 
information which is not resolved by the photodiodes, and thus appears as decoherence in the resulting
measurements\cite{kwiat00}.  A quarter wave plate and half wave plate before a polarizing beam splitter
in each beam path allow the polarization to be measured in any desired basis.  Avalanche photodiodes are
connected to coincidence detecting electronics to perform the final measurement.

To characterize the results of various entanglement detection schemes the processes described above are used to
create the set of states described by 
\( \rho(\theta) = 
\lambda(\cos (\theta) \ket{HH} + \sin (\theta) \ket{VV})(\cos (\theta) \bra{HH} + \sin (\theta) \bra{VV}) + 
(1 - \lambda)\ket{HV}\bra{HV} )\),
 or written out in the basis \(\ket{HV}\), \(\ket{HH}\), \(\ket{VV}\), \(\ket{VH}\):

\[ \left( \begin{array}{cccc}
1 - \lambda	& 0										& 0										& 0 \\
0			& \lambda\cos ^2 (\theta)				& \lambda\cos (\theta) \sin (\theta)	& 0 \\
0			& \lambda\cos (\theta) \sin (\theta)	& \lambda\sin ^2 (\theta)				& 0 \\ 
0			& 0										& 0										& 0 \\
\end{array} \right) \]

It is clear from evaluating \(Tr[\rho(\theta)^2]\) that this system is always mixed except in the trivial
case \(\lambda = 0, 1\).  Showing that a given density matrix is entangled is generally much harder, but
can be done by evaluating the eigenvalues of the partial transpose of \(\rho\)\cite{horodecki96}.  
These will be calculated later, and it can be shown that these states are entangled 
for \(0^{\circ} < \theta < 180^{\circ}.\)

Since the CHSH inequality is the most well-known of these measurements it will be used as a basis
of comparison for the others.  For this reason the value of \( \lambda \) is chosen such that the modified CHSH 
inequality given below has the value 0. These states have the maximum possible value of this inequality 
that can still be explained by local realism, and thus are not proven to be entangled by the test.

\section{CHSH and Bell Inequalitites}
Bell's inequality (in the CHSH form) states a limit for any state that is described by a local hidden variable
theory. Two measurement bases (1 and 2) in each of the branches of the experiment (A and B) are chosen, and the 
correlations between the measurements are bounded by 
\(C[A_1, B_1] + C[A_1, B_2] + C[A_2, B_1] - C[A_2, B_2] \leq 2\).  
In this original form, these correlations correspond to the expectation value of the product of the two 
measurements where the measured values are assumed to be +1 or -1.  In the context of polarization
optics and photodiodes, the above inequality can be shown to be equivalent to the more easily measured result
\[P_{A_1,B_1} + P_{A_1,B_2} + P_{A_2,B_1} - P_{A_2,B_2} - (P_{A_1} + P_{B_1}) \leq 0,\] where the P values
represent probabilities of coincidence under the relevant measurement settings.  The probabilities are
derived from raw coincidence rates by measuring a complete set of bases to estimate the total coincidence
rate, then dividing by this value and relying on the fair sampling assumption\cite{clauser69}.  The
probabilities that involve only one measurement can be calculated by summing probabilities
measured with orthogonal bases in the other arm or by removing the polarizer there.  Any state
which violates these inequalities cannot be described by local realism and obviously must be entangled, but
the converse is not necessarily true as the experimental results will clearly show.

All other inequalities that follow from local realism and require measurements in only two bases in the two
arms of the experiment can be shown to reduce to the CHSH inequality.  However, in the case where measurements
are taken in three bases an inequivalent inequality can be derived\cite{collins04}:
\[P_{A_1,B_1} + P_{A_2,B_1} + P_{A_3,B_1} + P_{A_1,B_2} + P_{A_2,B_2} + P_{A_1,B_3} -
P_{A_3,B_2} - P_{A_2,B_3} - P_{A_1} - 2P_{B_1} - P_{B_2} \leq 0.\]
This Geneva inequality correctly identifies entangled states that the CHSH inequality does not, and 
therefore clearly does not reduce to the CHSH inequality.

Detection bases must be determined before either of these quantities can be measured, and this choice may have 
optimal values for given input states.  For example, the well-known Bell test angles 0, 45, 22.5, and 67.5 degrees 
(for A1, A2, B1, B2 respectively) are optimal for the maximally entangled Bell states.  The
Bell test angles maximize violation of the tradition CHSH inequality with the value \(2\sqrt{2}\).  The 
optimal CHSH angles can be found analytically, but no such relationship has been discovered
for the Geneva inequality.

As shown in Fig. 2(a) \cite{altepeter05} the Geneva inequality proves that \( \rho(\theta) \) is entangled
for \(0 \leq \theta \leq 30^{\circ}\) and \(60^{\circ} \leq \theta \leq 90^{\circ}\).  This set of states was
chosen such that CHSH never detects its entanglement and the Geneva inequality also fails to do so near 
\(45^{\circ}\).  The measurement bases for the Geneva inequality were found by performing a random search of 
the three measurement angles while maximizing the inequality, so numerical imprecision likely accounts of the 
rounded behaviour of the predicted values near \(30^{\circ}\) and \(90^{\circ}\).  The experimental data is 
shown with errors bars demonstrating the bounds on the fidelity of the generated state.  Neither of these methods
is guaranteed to detect entanglement as shown, and neither quantifies the degree of entanglement present.

\section{Partial Transposition and Entanglement Witnesses}
Another method for detecting, but not quantizing entanglement is the entanglement witness.  The entanglement 
witness, W, is a Hermitian operator such that for any entangled state 
\(\ket{\alpha}\), \(\bra{\alpha}W\ket{\alpha} \geq 0\).  For the case of bipartite systems, the
entanglement witness operator can be derived from the partial transposition of the density matrix.  
The partial transpose with respect to the A measurements can be written as 
\( \rho^{T_A} = T(\rho) \otimes I \) which when acted on our density matrix produces
\( \rho^{T_A} \sum_{i,j,k,l} c_{k,l}^{i,j} \ket{A_i}\bra{A_j} \otimes \ket{B_k}\bra{B_l} =
\sum_{i,j,k,l} c_{k,l}^{i,j} \ket{A_j}\bra{A_i} \otimes \ket{B_k}\bra{B_l} \).  
Once again the determination of the relevant measurement (in this case the witness operator) depends 
on knowledge of the initial state, and thus is not directly applicable to arbitrary initial states. 
For the case of bipartite systems, it has been proven that a state is separable iff its partial transpose is a
positive operator\cite{horodecki96}. That is, for all projectors P: \[ Tr[\rho^{T_A} P] \geq 0 \]

For our general input state \(\rho(\theta)\), the partial transposition is given in the same basis as above
by:

\[ \left( \begin{array}{cccc}
1 - \lambda							& 0				& 0				& \lambda\cos (\theta) \sin (\theta)	\\
0									& \lambda\cos ^2 (\theta)		& 0										& 0 \\
0									& 0								& \lambda\sin ^2 (\theta)				& 0 \\ 
\lambda\cos (\theta) \sin (\theta)	& 0								& 0										& 0 \\
\end{array} \right) \]

The eigenvalues of this density matrix are \(\lambda \cos^2 (\theta), \lambda \sin^2 (\theta),
\frac{1}{4} (2 - 2\lambda \pm \sqrt{2} \sqrt{2 - 4\lambda + 3\lambda^2 - \lambda^2 \cos (4\theta)}.\)  For each 
value of \(\theta\), the minimum eigenvalue (\(\lambda_{min} (\theta)\)) is taken and the projector onto the 
corresponding eigenvector (\(\rho^{T_A}\ket{e(\theta)}\bra{e(\theta)} = \ket{e(\theta)}\bra{e(\theta)}^{T_A} \)) 
serves as an entanglement witness.  It is clear that measuring
the projection of \(\rho^{T_A}\) onto this ket will minimize \( Tr[\rho^{T_A} P] \).  Further, it can be shown
that \(\ket{e(\theta)}\) is independent of \(\theta\) and can be written \(\ket{e}.\)  
In Fig. 2(b) \cite{altepeter05} it is shown that this entanglement witness correctly predicts entanglement
for this set of input states.

For the set of input states chosen in this experiment, the Hefei inequality gives the same results as
the entanglement witness experiment.  The Hefei inequality states that for two sets of three mutually
complementary observables \(\{A_i\}\) and \(\{B_i\}\) such that \(A_1A_2A_3 = B_1B_2B_3\), a bipartite state
is separable iff\cite{yu03} 
\[ \sqrt{Tr[\rho(A_1B_1 + A_2B_2)]^2 + Tr[\rho(A_3 + B_3)]^2} - Tr[\rho A_3B_3] \leq 1. \]
In this case the left hand side can be shown to be a function of \(\lambda_{min} (\theta)\) and once 
again entanglement is correctly
predicted.  The Hefei inequality is no more powerful than the entanglement witness, and given the correct
choices of measurements both can determine entanglement conclusively.  Both are also unable to quantify
the amount of entanglement present.

\section{Conclusions}
None of the methods presented here are particularly compelling.  They all require previous knowledge of the state of
the system, have not been shown to generalize well to higher dimensional systems, and do not quantify the degree 
of entanglement
present in the system.  Altepeter, et al. conclude that the best option is to perform the full quantum state 
tomography\cite{altepeter05}, but this requires measurement in many different bases and as discussed above 
scales exponentially to higher dimensional systems.  Of the options presented, the entanglement witness and
Hefei inequality
provides the most conclusive information and are relatively easy to measure in bipartite states.  
As future developments are made in quantum computing, and the degree of complexity in quantum systems grows, 
new tools to analyze them will be needed.

\begin{thebibliography}{99}
\bibitem{altepeter05} J.B. Altepeter \emph{et al.}, Phys. Rev. Lett. 95, 033601 (2005).

\bibitem{kwiat00} P. G. Kwiat, A. J. Berglund, J. B. Altepeter, and A. G. White, 
	"Experimental verification of decoherence-free subspaces", Science 290, 498 (2000).
	
\bibitem{clauser69} J. F. Clauser, M.A. Horne, A. Shimony, and R. A. Holt,
	Phys. Rev. Lett. 23, 880 (1969).
	
\bibitem{collins04} D. Collins and N. Gisin, J. Phys. A 37, 1775 (2004).

\bibitem{horodecki96} M. Horodecki, P. Horodecki, and R. Horodecki, Phys. Lett. A 223, 1 (1996).

\bibitem{yu03} S. Yu, J.-W. Pan, Z.-B. Chen, and Y.-D. Zhang, Phys. Rev.
	Lett. 91, 217903 (2003).
\end{thebibliography}

\end{document}

